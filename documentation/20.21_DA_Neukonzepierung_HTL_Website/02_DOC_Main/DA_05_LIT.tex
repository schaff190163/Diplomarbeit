%Literaturverzeichnis: Alle Zitate und Veröffentlichungen, die bekannt sind finden sich in diesem Verzeichnis.
%Nur diejenigen, die im Dokument auch verwendet werden kommen in die Liste der Compilierung 
%
%usage im Dokument: \cite{Kurzzeichen}   hier wird im Dokument nur der Text in eckiger Klammer abgebildet
%usage im Dokument: \cite[S~105]{11_Killinger} hier wird auch eine Seitenanzahl dazugestellt
\begin{thebibliography}{22}
%
% Einträge für Literaturverzeichnis:
%usage im Verzeichnis: \bibitem[Werk, Jahrgang oder Auflage]{Kurzzeichen}{Text der als Erklärung im Glossar steht!}
%
   \bibitem[Killinger, 1998]{11_Killinger} Sprache heute : Schriftverkehr; Deutsch für berufsbildende Schulen,\newline 
Seit der Einführung der Bildungsstandards steht der Erwerb fachlicher und sozialer Kompetenzen für Alltag und Beruf im Fokus des Unterrichts.\newline
Autoren: Walter Pirnath, Robert Killinger, Josef Neumüller, 1998
%
\bibitem[vgl. Mustermann, 2020]{12_MM} Skriptum: Test im Zuammenhang mit  ...\newline 
Autoren: Max Mustermann, 2020
%
\bibitem[Werk 1, Jahrgang oder Auflage 1]{BlindzitatKurzzeichen_1}{Text der als Erklärung im Literaturverzeichnis steht!}
%
% weitere Einträge ...
%
\end{thebibliography}
