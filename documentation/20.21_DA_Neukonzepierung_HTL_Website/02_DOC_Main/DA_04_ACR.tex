%Abkürzungsverzeichnis: Alle Abkürzungen, die bekannt sind finden sich in diesem Verzeichnis.
%Nur diejenigen, die im Dokument auch verwendet werden kommen in die Liste der Compilierung 
%
%der Name der Überschrift kann frei gewählt werden (in diesem Fall: Abkürzungen):
%
%\section*{Acronymverzeichnis}					%Alternative Überschrift für das Verzeichnis
\section*{Abkürzungen}
%
%usage im Dokument: \acs{DA}
%usage im Dokument: \acs{Kurzzeichen}
\begin{acronym}[längstesAcronym]
\setlength{\itemsep}{-0.3 \parsep}
%
% Einträge für Abkürzungen:
%usage im Verzeichnis: \acro{Kurzzeichen}[Inhalt der im Text steht]{Text der als Erklärung im Glossar steht!}
%
\acro{DA}[Diplomarbeit]{Text für die Erklärung der Diplomarbeit}
\acro{ESD}[ESD]{Electrostatic Discharge}
\acro{EMC}[EMC]{Electromagnetic Compatibility}
%
%
% weitere Einträge ...
%
\end{acronym}
