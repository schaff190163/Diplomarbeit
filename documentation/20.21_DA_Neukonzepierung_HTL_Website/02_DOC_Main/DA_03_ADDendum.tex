% Zusammenfassung und Resumee der Diplomarbeit:
\section{Zusammenfassung}

Hier steht die Zusammenfassung und das Resumee der Diplomarbeit!
\newpage

% Anhang zur Diplomarbeit:
\section{Anhang}

Hier stehen zusätzliche Informationen zur Diplomarbeit!
Zum Beispiel können hier Listings oder Teile des Programmes abgebildet sein:

\begin{lstlisting}
// create new document
string path = @"C:\Data\sample.xlsx";
// add title row 
List<string> titleRow = new List<string>();
titleRow.Add("This is the 1st cell");
titleRow.Add("This is the 2nd cell");
openXmlExcel.addTitleRow(titleRow);
// add scatter chart
List<double[]> xValues = new List<double[]>();
List<double[]> yValues = new List<double[]>();
List<string> lineNames = new List<string>();
// first line
xValues.Add(new double[5] { -2, -1, 0, 1, 2 });
yValues.Add(new double[5] { -2, -1, 0, 1, 2 });
lineNames.Add("Line 1");
// second line
xValues.Add(new double[5] { -2, -1, 0, 1, 2 });
yValues.Add(new double[5] { -3, -1, 0, 1, 3 });
lineNames.Add("Line 2");
// save document 
openXmlExcel.saveFile();
\end{lstlisting}

\newpage

\section{Verzeichnisse}
% Bildverzeichnis:
\listoffigures 
%
% Abkürzungsverzeichnis:
%Abkürzungsverzeichnis: Alle Abkürzungen, die bekannt sind finden sich in diesem Verzeichnis.
%Nur diejenigen, die im Dokument auch verwendet werden kommen in die Liste der Compilierung 
%
%der Name der Überschrift kann frei gewählt werden (in diesem Fall: Abkürzungen):
%
%\section*{Acronymverzeichnis}					%Alternative Überschrift für das Verzeichnis
\section*{Abkürzungen}
%
%usage im Dokument: \acs{DA}
%usage im Dokument: \acs{Kurzzeichen}
\begin{acronym}[längstesAcronym]
\setlength{\itemsep}{-0.3 \parsep}
%
% Einträge für Abkürzungen:
%usage im Verzeichnis: \acro{Kurzzeichen}[Inhalt der im Text steht]{Text der als Erklärung im Glossar steht!}
%
\acro{DA}[Diplomarbeit]{Text für die Erklärung der Diplomarbeit}
\acro{ESD}[ESD]{Electrostatic Discharge}
\acro{EMC}[EMC]{Electromagnetic Compatibility}
%
%
% weitere Einträge ...
%
\end{acronym}

%
% Literaturverzeichnis:
%Literaturverzeichnis: Alle Zitate und Veröffentlichungen, die bekannt sind finden sich in diesem Verzeichnis.
%Nur diejenigen, die im Dokument auch verwendet werden kommen in die Liste der Compilierung 
%
%usage im Dokument: \cite{Kurzzeichen}   hier wird im Dokument nur der Text in eckiger Klammer abgebildet
%usage im Dokument: \cite[S~105]{11_Killinger} hier wird auch eine Seitenanzahl dazugestellt
\begin{thebibliography}{22}
%
% Einträge für Literaturverzeichnis:
%usage im Verzeichnis: \bibitem[Werk, Jahrgang oder Auflage]{Kurzzeichen}{Text der als Erklärung im Glossar steht!}
%
   \bibitem[Killinger, 1998]{11_Killinger} Sprache heute : Schriftverkehr; Deutsch für berufsbildende Schulen,\newline 
Seit der Einführung der Bildungsstandards steht der Erwerb fachlicher und sozialer Kompetenzen für Alltag und Beruf im Fokus des Unterrichts.\newline
Autoren: Walter Pirnath, Robert Killinger, Josef Neumüller, 1998
%
\bibitem[vgl. Mustermann, 2020]{12_MM} Skriptum: Test im Zuammenhang mit  ...\newline 
Autoren: Max Mustermann, 2020
%
\bibitem[Werk 1, Jahrgang oder Auflage 1]{BlindzitatKurzzeichen_1}{Text der als Erklärung im Literaturverzeichnis steht!}
%
% weitere Einträge ...
%
\end{thebibliography}
